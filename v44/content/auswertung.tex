\section{Auswertung}
\label{sec:Auswertung}

Um auf die Eigenschaften des Siliziumwafers schließen zu können, muss das Messgerät zunächst kalibriert und die hinsichtlich dessen ausgewertet werden.
Die Auswertung erfolgt mithilfe der python Bibliotheken numpy \cite{numpy}, scipy \cite{scipy} und uncertainties \cite{uncertainties}, während die Grafiken mit matplotlib \cite{matplotlib} erstellt werden.

\subsection{Bestimmung der Halbwertsbreite und der maximalen Intensität}
\label{subsec:Halbwertsbreite}

Um die Halbwertsbreite und auch die maximale Intensität zu bestimmen, wird an die Messdaten eines Detectorscans eine Gaußfunktion der Form aus Gleichung (\ref{eq:Gauss}) gefittet. 
Dabei ist $I(\alpha)$ die Intensität in Abhängigkeit des Winkels $\alpha$, $I_0$ die maximale Intensität, $\alpha_0$ der Winkel bei dem die maximale Intensität auftritt, $\sigma$ die Halbwertsbreite und $B$ der Untergrund.
Eine graphische Darstellung der Messdaten und des Fits ist in Abbildung \ref{fig:Gauss} zu sehen.
\begin{align}
    I(\alpha) = \frac{I_0}{\sqrt{2\pi\sigma^2}} \exp\left(-\frac{(\alpha-\alpha_0)^2}{2\sigma^2}\right) +B
    \label{eq:Gauss}
\end{align}
\begin{figure}[H]
    \centering
    \includegraphics[width=0.8\textwidth]{./build/DetectorScan.pdf}
    \caption{Messdaten des Detectorscans und Fit zur Bestimmung der Halbwertsbreite und der maximalen Intensität.}
    \label{fig:Gauss}
\end{figure}
\noindent
Die freien Parameter des Fits ergeben sich zu
\begin{align*}
    \alpha_0 &= \SI{2.6767(0.4602)e-3}{\degree}, \\
    \sigma &= \SI{3.6969(0.0041)e-2}{\degree}, \\
    I_0 &= \SI{3.4247(0.0411)e4},\\
    B &= \SI{5.1979(0.9178)e4}.\\
\end{align*}
Die Halbwertsbreite entspricht der Breite der Funktion an der die Intensität auf halber Höhe ist, weshalb man sie auch mit \textit{FWMH} (full width at half maximum) bezeichnet.
Sie ergibt sich über $\text{FWHM} = 2\sqrt{2\ln(2)}\sigma$ zu
\begin{align*}
    \text{FWHM} = \SI{0.0871(0.0011)}{\degree}.
\end{align*}


\subsection{Bestimmung der Strahlbreite}
\label{subsec:Strahlbreite}

Um die Strahlbreite zu bestimmen, wird ein Z-Scan durchgeführt.
Da in diesem Test die z-Achse der Probe in den Strahlengang bewegt wird, der zunächst vollständig auf den Detektor fällt, wird die Intensität in Abhängigkeit der Position der Probe gemessen.
Der Abstand, der benötigt wird, damit die Intensität vollständig auf null fällt, entspricht der Strahlbreite.
Eine grafische Auswertung des Z-Scans ist in Abbildung \ref{fig:ZScan} zu sehen.
\begin{figure}[H]
    \centering
    \includegraphics[width=0.8\textwidth]{./build//ZScan.pdf}
    \caption{Messdaten des Z-Scans zur Bestimmung der Strahlbreite.}
    \label{fig:ZScan}
\end{figure}
\noindent
Die Strahlbreite bestimmt sich zu $d_0\approx\SI{0.2}{\milli\meter}$.

\subsection{Bestimmung des Geometriewinkels}
\label{subsec:Geometriewinkel}

Um den in \autoref{subsec:Röntgenröhre} beschriebenen systematischen Fehler zu beheben, wird der Geometriewinkel bestimmt.
Dazu wird ein Rockingscan durchgeführt, woraus sich der Geometriewinkel mithilfe der Breite der Messwerte bestimmt.
Die Messdaten und die Breite, die zu dem Geometriewinkel führt, sind in Abbildung \ref{fig:RockingScan} zu sehen.
\begin{figure}[H]
    \centering
    \includegraphics[width=0.8\textwidth]{./build//RockingScan.pdf}
    \caption{Messdaten des Rockingscans zur Bestimmung des Geometriewinkels.}
    \label{fig:RockingScan}
\end{figure}
\noindent
Der Geometriewinkel ergibt sich mithilfe von Formel (\ref{eq:Geometriefaktor}) zu $\alpha_{\text G} = \SI{0.57}{\degree}$.

\subsection{Auswertung der Dispersion und der Rauigkeit des Siliziumwafers}
\label{subsec:Dispersion}

Für die Auswertung des mit Polysterol beschichteten Siliziumwafers und dessen Oberflächenrauigkeit und auch die Schichtdicken zu bestimmen, werden Reflektivitätsscans durchgeführt.
Zunächst werden ein normaler Reflektivitätsscan und ein Diffuser-Scan durchgeführt, um die Messdaten von dem Untergrund zu bereinigen.
Eine grafische Auswertung der Messdaten und der Differenz der beiden Scans ist in Abbildung \ref{fig:Reflektivität} zu sehen.
\begin{figure}[H]
    \centering
    \includegraphics[width=0.8\textwidth]{./build//ReflectDiffuserScan.pdf}
    \caption{Messdaten des Reflektivitäts- und Diffusorscans und die Differenz der beiden Scans.}
    \label{fig:Reflektivität}
\end{figure}
\noindent
In den nächsten Scans wird die Messzeit pro Messpunkt auf $\SI{5}{\second}$ erhöht, weshalb die Reflektivität $R$ betrachtet wird, die sich über
\begin{align*}
    R = \frac{I_{\text{Probe}}}{5\cdot I_{0}}.
\end{align*}
Es wird zusätzlich der Geometriefaktor aus Formel (\ref{eq:Geometriefaktor}) berücksichtigt.
Der Verlauf der Reflektivität einer idealen Probe ergibt sich über die Fresnelreflektivität nach Gleichung (\ref{eq:Fresnel}) %$\left(\frac{\alpha_{\text{crit}}}{2\cdot\alpha}\right)^4$, 
wobei sich der kritische Winkel der Probe über Gleichung (\ref{eq:alpha_C}) zu $\alpha_{\text{crit}} = \SI{0.223}{\degree}$ ergibt.
Mithilfe der Wellenlänge der $K_{\alpha}$-Linie von Kupfer $\lambda = \SI{1.54}{\angstrom}$ und der periodischen Minima der Kiessing-Oszillationen, lässt sich die Schichtdicke der Probe bestimmen.
Sie sind gemeinsam mit der korriegierten und unkorriegierten Reflektivität und der Fresnelreflektivität in Abbildung \ref{fig:Reflektivität2} dargestellt.
\begin{figure}[H]
    \centering
    \includegraphics[width=0.8\textwidth]{./build//Reflectivity.pdf}
    \caption{(Un-)Bereinigte Reflektivität und die Fresnelreflektivität.}
    \label{fig:Reflektivität2}
\end{figure}
\noindent
Die Schichtdicke der Probe ergibt mit $\Delta\alpha= \SI{5.12(0.49)e-2}{\degree}$ zu $d = \SI{8.62(0.82)e-8}{\meter}$. \newline
Mithilfe des Parrattalgorithmus aus Gleichung (\ref{eq:Parratt}) eines Mehrschichtsystems, lassen sich die Dispersion $\delta$, die Rauigkeit $\sigma$ und die Dicke der Polysterolschicht bestimmen.
Die mit dem Parrattalgorithmus berechnete Fit ist mit der bereinigten Reflektivität in Abbildung \ref{fig:Reflektivität3} dargestellt.
\begin{figure}[H]
    \centering
    \includegraphics[width=0.8\textwidth]{./build//Parratt.pdf}
    \caption{Bereinigte Reflektivität und Fit des Parrattalgorithmus.}
    \label{fig:Reflektivität3}
\end{figure}
\noindent
Die Startwerte der freien Parameter wurde auf aus der Literatur bekannten Werte gesetzt.
Die freien Parameter des Fits ergeben sich zu
\begin{align*}
    &\delta_{\text{Si}} = \SI{1.67e-6}, & & \delta_{\text{Poly}} = \SI{8.89e-6}, \\
    &\beta_{\text{Si}} = \SI{2.41e-8}, & &\beta_{\text{Poly}} = \SI{5.05e-10}, \\
    &\sigma_{\text{Luft, Poly}} = \SI{3.59e-11}, & & \sigma_{\text{Poly, SI}} = \SI{7.94e-8}, \\
    &\alpha_{\text{c, Poly}} = \SI{0.27}{\degree}, & & \alpha_{\text{c, SI}} = \SI{0.073(0.006)}{\degree}, 
    % \\
    % & &d_{\text{Poly}} = \SI{1.0e-6}.
\end{align*}
\begin{align*}
    d_{\text{Poly}} = \SI{1.21e-7}.
\end{align*}