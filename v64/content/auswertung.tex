\section{Analysis}
\label{sec:Analysis}

The following analysis is performed with the programming language python \cite{python} and the libraries numpy \cite{numpy} for numerical calculation, scipy \cite{scipy} for curve fitting and uncertainties \cite{uncertainties} for error calculation.
Furthermore, the graphical representations are created with the matplotlib \cite{matplotlib} library.

\subsection{Dependence of the contrast on the polarisation angle}
\label{subsec:polarisation}

The first analysis step is to determine the dependence of the contrast on the polarisation angle.
For this, the maximum and minimum voltage of the photodiodes was measured in dependence of the polarisation angle.
The original measurement values are shown in figure \ref{fig:original_data} in appendix \ref{sec:appendix}.
The mean values of the measurements and the with equation (\ref{eq:contrast}) calculated contrast is shown in table \ref{tab:values_polarisation}.

\begin{table}[H]
    \centering
    \caption{Measured Values of the contrast in dependence on the polarisation angle.}
    \label{tab:values_polarisation}
    \begin{tabular}{c c c c}
        \toprule
        $\phi \,/\, \si{\degree}$ & $I_{\text{max}}\,/\,\si{\volt}$ & $I_{\text{min}}\,/\,\si{\volt}$ & contrast \\
        \midrule
        0.0 & $4.98$ & $4.52\pm0.01$ & $0.05$ \\
        15.0 & $4.37\pm0.02$ & $2.35\pm0.01$ & $0.30$ \\
        30.0 & $3.91$ & $1.15\pm0.01$ & $0.55$ \\
        45.0 & $4.45$ & $0.81$ & $0.69$ \\
        60.0 & $5.86\pm0.01$ & $1.07\pm0.01$ & $0.69$ \\
        75.0 & $6.53\pm0.05$ & $2.29\pm0.01$ & $0.48$ \\
        90.0 & $7.10\pm0.05$ & $5.38\pm0.04$ & $0.14$ \\
        105.0 & $9.92\pm0.11$ & $4.42\pm0.07$ & $0.38\pm0.01$ \\
        120.0 & $13.15\pm0.13$ & $2.71\pm0.04$ & $0.66$ \\
        135.0 & $17.61\pm1.42$ & $1.65$ & $0.83\pm0.01$ \\
        150.0 & $12.94\pm0.10$ & $2.53\pm0.01$ & $0.67$ \\
        165.0 & $9.87\pm0.08$ & $3.86\pm0.03$ & $0.44$ \\
        180.0 & $5.33\pm0.04$ & $4.78\pm0.01$ & $0.05$ \\
        \bottomrule
    \end{tabular}
\end{table}

\noindent
A graphical representation of the contrast in dependence on the polarisation angle is shown in figure \ref{fig:contrast}, while a theroetical prediction of the form
\begin{align}
    K = K_0\cdot|\sin^2(\phi-\kappa)|
\end{align}
is fitted to the data.\\
The offset of $\delta$ that is used to compensate for deviations of the experimental setup and the amplitude $K_0$ are determined to be
\begin{align*}
    K_0=\SI{0.697\pm0.004}{} \quad \text{and} \quad \kappa=\SI{1.620\pm0.010}{\degree}.
\end{align*}
For the following measurements, the polarisation angle is set to $\SI{135}{\degree}$, as this angle provides the highest contrast.
\begin{figure}[H]
    \centering
    \includegraphics[width=0.75\textwidth]{build/contrast.pdf}
    \caption{Graphical representation of the contrast in dependence on the polarisation angle.}
    \label{fig:contrast}
\end{figure}

\subsection{Refraction index of glass}
\label{subsec:refraction_glass}

In order to analyse the refractive index of the glass, a total of five series of measurements were carried out in which the angle of the glass plate was varied by $\SI{10}{\degree}$.
The measurements of the refraction index of glass are shown in table \ref{tab:refraction_glass}, giving a mean value of $\overline M=\num{33.4\pm1.4}$.
\begin{table}[H]
    \centering
    \begin{tabular}{c c c}
        \toprule
        Measurement series & $n$ \\
        \midrule
        1 & 31 \\
        2 & 34 \\
        3 & 34 \\
        4 & 33 \\
        5 & 35 \\
        \bottomrule
    \end{tabular}
    \caption{Measured counts for the refraction index of glass for different measurement series with $\upDelta \vartheta=\SI{10}{\degree}$.}
    \label{tab:refraction_glass}
\end{table}
\noindent
To calculate the refraction index of glass, equation (\ref{eq:interference_maxima}) and equation (\ref{eq:solid}) is simplified to the form
\begin{align}
    n = \frac{1}{1-\frac{M\lambda_{\text{vacuum}}}{2 L\theta^2}}
\end{align}
where $d$ denotes the thickness of the glass plate, $\lambda$ the wavelength of the light in vacuum and $M$ the measured maxima.
The refraction index of glass is then calculated to be $n=\num{1.531 \pm 0.034}$.

\subsection{Refraction index of air}
\label{subsec:refraction_air}

The refraction index of air is determined by measuring the number of maxima in dependence of the pressure.
The original measurement values are shown in figure \ref{fig:original_data} in appendix \ref{sec:appendix}, while the mean values of the 3 measurement series are shown in table \ref{tab:refraction_air}.
\begin{table}[H]
    \centering
    \begin{tabular}{c c}
        \toprule
        pressure / $\si{\milli\bar}$  & counts \\    
        \midrule
        50.0 & $2.0$\\
        100.0 & $4.0$\\
        150.0 & $6.0$\\
        200.0 & $8.0$\\
        250.0 & $10.0$\\
        300.0 & $12.0$\\
        350.0 & $14.0$\\
        400.0 & $16.0$\\
        450.0 & $18.5\pm0.6$\\
        500.0 & $21.0$\\
        550.0 & $23.0$\\
        600.0 & $25.0$\\
        650.0 & $27.0$\\
        700.0 & $29.0$\\
        750.0 & $31.0\pm0.6$\\
        800.0 & $33.0\pm0.6$\\
        850.0 & $36.0$\\
        900.0 & $38.0$\\
        950.0 & $40.0$\\
        1000.0 & $42.0$\\
    \end{tabular}
    \caption{Measured counts for the refraction index of air in dependence of the pressure.}
    \label{tab:refraction_air}
\end{table}
\noindent
The refraction indices are then calculated using the formula
\begin{align}
    n_{\text{air}} = 1 + \frac{M\cdot\lambda}{L},
\end{align}
where $L$ denotes the length of the air cell and $\lambda$ the wavelength of the light in vacuum.
The calculated values are shown in figure \ref{fig:refraction_index}.
The Lorentz-Lorenz law from equation (\ref{eq:Lorentz-Lorenz}) can bew simplified for $n\approx1$ to the form
\begin{align}
    n = \frac 32 \frac{Ap}{RT}+1.
\end{align}
This way the refraction index of air can be determined via a linear fit to the data of the form
\begin{align}
    n(p, T\equiv T_0=\SI{288.15}{\kelvin}) = \frac 32 \frac{p}{R\cdot T_0}\cdot a + b,
\end{align}
with the free parameters $a$ and $b$.
The fit yields the parameters
\begin{align*}
    a = \SI{0\pm 10e-5}{\per\milli\bar} \quad \text{and} \quad b = 1 - \num{7\pm5e-7}.
\end{align*}
Thus, the experimental value for the refraction index of air is $n_{\text{air, exp}}=1-\num{7\pm5e-7}$.
Similarly, the refraction index of air at standard conditions, namely an atmospheric pressure of $p_0=\SI{1013.25}{\milli\bar}$ and a room temperature of $T_0=\SI{288.15}{\kelvin}$, is calculated using the same measurements to be $n_{\text{air, standard}}=1\pm\num{6e-5}$.
\begin{figure}[H]
    \centering
    \includegraphics[width=0.75\textwidth]{build/refraction_index.pdf}
    \caption{Graphical representation of the refraction index of air in dependence of the pressure.}
    \label{fig:refraction_index}
\end{figure}