\section{Discussion}
\label{sec:Discussion}

The dependence of the polarisation angle of the contrast has shown a sufficient agreement between the expected theory and the measured values with an angle-offset of $\delta=\SI{1.620\pm 0.010}{\degree}$.
This offset, while not exceedingly substantial, is still noteworthy and the main source of error in this experiment.
Further sources of errors are small misalignments that add up over the course of the path of the light.
This is reflected in the maximum contrast that was measured to be $K= \SI{0.697\pm 4e-3}{}$, while the theoretical maximum is $K=1$.
It results in a deviation of $\SI{30.32\pm0.4}{\percent}$. 
This deviation indicates that the measurement configuration was not perfectly aligned and that a more precise alignment could have improved the results. \newline
The refraction index of glass was determined to be $n=\num{1.531 \pm 0.034}$, which is in good agreement with the literature value of $n_{\text{literature}}(\lambda=\SI{632.99}{\nano\meter})=\num{1.52}$ \cite{RefractiveIndex}.
Resulting in a deviation of $\SI{5.11 (2.3)}{\percent}$.
Counting in the fact that the setup was not perfectly aligned, this deviation is satisfactory.
The refraction index of air was determined to be $n=1-\num{7 \pm 5 e-7}$, which is in perfect agreement with the literature value of $n_{\text{air}}=\num{1.00027653}$ \cite{RefractiveIndex} as it shows a deviation of $\SI{7e-5}{\percent}$.
Furthermore, the refraction index of air at standard conditions was calculated using the values of the measurement to be $n_{\text{air, standard}}=1\pm\num{6e-5}$ which deviates by $\SI{0.028 (0.006)}{\percent}$ from the literature value. \newline
In summary, the setup was sufficiently well aligned, although a more thorough investigation into the setup could have improved the resulting measurements.
Regardless of the limitations of the setup, all of the different refractive indices that were calculated can be seen as a success, as they all show small deviations from the literature values.