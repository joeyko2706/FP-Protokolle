\section{Experimental setup and procedure}
\label{sec:Durchführung}
The Sagnac interferometer has been set up as described in \autoref{sec:The_Sagnac_Interferometer}. The experiment had to be adjusted carefully by tilting the mirrors until the beams were completely aligned. The next step of adjustment required putting a polarisation filter in the overlapping beam leaving the interferometer. This had to be done because without it, their polarisations were perpendicular to each other, thus no interference could happen. The change in polarisation resulted in an interference pattern becoming visible, because the beams were not perfectly aligned, resulted in slightly different path lengths and by that a phase difference between the two. Fine-tuning was then done by further adjusting the device until the interference fringes disappeared. This part of adjustment was the final step in perfecvtly aligning the beams.\\ 
After adjustment, the special property of the Sagnac interferometer was used: This interfermeter makes it possible to slightly move the mirror in front of the PBSC, as was also described in \autoref{sec:The_Sagnac_Interferometer}.\\\\
Installation of the rotation holder inserted the glass plates (thickness $1 \text{mm}$) into both beams, and the interference pattern as a function of the rotation angle has been measured in steps from $-1\textdegree$ to $9\textdegree$ in steps of $1\textdegree$.\\\\
The polarisation filter then was replaced by a $45\textdegree$ PBSC, directing the two beams into a photo diod each, which were coupled to an oscilloscope to display their intensity as a function of time.\\
\\
As a final step, the experiment was placed under a plexiglass hood to minimise air density fluctuations, and thus the background.\\
\\
To measure the contrast of the interferometer, the rotation holder with the glass plates was rotated. Doing so resulted in the diodes showing minima and maxima in voltage, which are proportional to the intensity at minimum/maximum. The minimum and maximum intensity was measured in a range of $0\textdegree$ to $180\textdegree$ in steps of $15\textdegree$ on the polarisation filter. The results were then used to set the polariser to maximum contrast.\\
\\
The next measurement's goal was to get the refractive index of glass. For this, the number of minima/maxima, depending on the rotation angle of the glass holder, was measured. For this, the glass holder was rotated slowly from $-1\textdegree$ to $9\textdegree$, with the photodiodes coupled to a counting device that counted the number of times that their output went through zero. For this it was important to do it slowly but steadily without stopping, since stopping close to zero could lead to counts generated by noise sending voltage through its zero value. This measurement was done $10$ times.\\
\\
In the final measurement, the rotation holder was removed. Instead, a gas cell (length $(100\pm 0.1)\text{mm}$ had been placed in one of the beams. The room was held at a constant temperature of $T = \text{Hier Temperatur einf[gen} \textdegree\text{C}$.\\
The cell was then evacuated, and the number of interference maxima was measured in steps of $50 \text{mbar}$ from $50\text{mbar}$ to $1000\text{mbar}$. This was performed three times.
