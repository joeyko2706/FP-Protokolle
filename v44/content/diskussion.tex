\section{Diskussion}
\label{sec:Diskussion}

Bei der Betrachtung des Detectorscans und der damit im Zusammenhang betrachteten Halbwertsbreite des Röntgenstrahls, die in \autoref{fig:Gauss} abgebildet wurden, wurde eine gaußförmige Verteilung an die Messdaten angelegt.
Es lässt sich feststellen, dass die Gaußfunktion die Messdaten sehr gut beschreibt und diese sehr symmetrisch um den Ursprung bei $\alpha=\SI{0}{\degree}$ verläuft.
Lediglich in den Bereichen des stärksten Anstiegs und Abfalls der Messdaten weicht die Gaußfunktion von den Messdaten leicht ab. \newline
In einem nächsten Schritt wurde ein Z-Scan durchgeführt, um die Strahlbreite des Röntgenstrahls zu bestimmen.
Eine grafische Darstellung der Messdaten und der eingezeichneten Strahlbreite ist in \autoref{fig:ZScan} zu sehen.
Diese wurde durch den charakteristischen Abfall der Intensität in Abhängigkeit der Position der Probe auf $d\approx\SI{0.2}{\milli\meter}$ bestimmt.
Um in diesem Schritt die Unsicherheit auf der festgestellten Strahlbreite zu verringern, könnte die Schrittweite pro Messschritt verringert werden. \newline
Der Geometriewinkel wurde durch einen Rockingscan bestimmt, der in \autoref{fig:RockingScan} abgebildet ist und ergab sich zu $\alpha_{\text{G}}=\SI{0.014}{\degree}$ weicht von dem theoretischen Wert von $\alpha_{\text{G,\text{Theorie}}}=\SI{0.01}{\degree}$ um $\SI{71.43}{\percent}$ ab.
Durch die Abhängigkeit des theoretischen Geometriewinkels von dem zuvor gemessenen Strahlbreite, könnte die Unsicherheit auf den Geometriewinkel durch eine genauere Bestimmung der Strahlbreite verringert werden und somit näher an dem experimentellen Wert liegen.
Aus diesem Grund wurde der experimentelle Wert für fortlaufende Auswertungen verwendet. \newline
Nach den Auswertung der Justierungen folgte die Auswertung der Dispersion und der Rauigkeit des Siliziumwafers.
Die in Reflektivitätsscans gemessenen Daten wurden in \autoref{fig:Reflektivität}, \autoref{fig:Reflektivität2} und \autoref{fig:Reflektivität3} dargestellt.
Sie zeigen klar zu erkennende Kiessing-Oszillation, die auf eine Schichtdicke von $d = \SI{8.62(0.75)e-8}{\meter}$ rückschließen ließen.\newline
Der Parratt-Algorithmus wurde verwendet, um die Dispersion und die Rauigkeit des Siliziumwafers zu bestimmen.
Er zeigt eine grobe Übereinstimmung mit den mit dem Korrekturfaktor bereinigten Messdaten, die in \autoref{fig:Reflektivität3} dargestellt sind. 
Die aus dem Parrattalgorithmus bestimmten Parameter sind gemeinsam mit den Literaturwerten und den relativen Abweichungen in \autoref{tab:parratt} aufgeführt.

% \begin{table}[h!]
%     \centering
%     \begin{tabular}{c c}
%         \toprule
%         \textbf{Parameter der Gaußfunktion} & \textbf{Wert} \\ 
%         \midrule
%         $\alpha_0$ & $\SI{2.6767(0.4602)e-3}{\degree}$ \\ 
%         $\sigma$ & $\SI{3.6969(0.0041)e-2}{\degree}$ \\ 
%         $I_0$ & $\SI{3.4247(0.0411)e4}{}$,\\
%         $B$ & $\SI{5.1979(0.9718)e4}{}$ \\ 
%         FWHM & $\SI{0.0871(0.0011)}{}$ \\ 
%         \bottomrule
%     \end{tabular}
%     \caption{Parameter der Gaußfunktion}
%     \label{tab:gauss}
% \end{table}

% \begin{table}[h!]
%     \centering
%     \begin{tabular}{c c}
%         \toprule
%         \textbf{Geometriewinkel} & \textbf{Wert} \\ 
%         \midrule
%         Experiment & $\SI{0.014}{\degree}$ \\ 
%         Theorie & $\SI{0.01}{\degree}$ \\ 
%         \bottomrule
%     \end{tabular}
%     \caption{Geometriewinkel}
%     \label{tab:geometriewinkel}
% \end{table}

% \begin{table}[h!]
%     \centering
%     \begin{tabular}{c c}
%         \toprule
%         \textbf{Schichtdicke} & \textbf{Wert} \\ 
%         \midrule
%         $\upDelta \alpha$ & $\SI{5.12(0.45)e-02}{}$ \\ 
%         Schichtdicke & $\SI{8.62 (0.75)e-08}{}$ \\ 
%         \bottomrule
%     \end{tabular}
%     \caption{Schichtdicke}
\begin{table}[h!]
    \centering
    \begin{tabular}{c c c c}
        \toprule
        \textbf{Parameter} & \textbf{Experiment} & \textbf{Literaturwert} & \textbf{Relative Abweichung} \\
        \midrule
        $\delta_{\text{Silizium}}$      & $\SI{9.53e-06}{}$         & $\SI{7.6e-06}{}$         & $\SI{25.39}{\percent}$ \\ 
        $\delta_{\text{Poly}}$          & $\SI{8.44e-06}{}$         & $\SI{3.5e-06}{}$         & $\SI{141.14}{\percent}$ \\ 
        $\alpha_{c,\text{Sili}}$        & $\SI{0.0047}{\degree}$    & $\SI{0.174}{\degree}$    & $\SI{97.30}{\percent}$ \\ 
        $\alpha_{c,\text{Poly}}$        & $\SI{0.0013}{\degree}$    & $\SI{0.153}{\degree}$    & $\SI{99.15}{\percent}$ \\ 
        \bottomrule
    \end{tabular}
    \caption{Parratt-Algorithmus}
    \label{tab:parratt}
\end{table}
\noindent
Die Schichtdicke wurde durch den Parrattalgorithmus zu $d = \SI{2.40e-08}{\meter}$ bestimmt und weicht somit um $\SI{72.17}{\percent}$ von dem im Reflektivitätsscan bestimmten Wert $d = \SI{8.62(0.75)e-8}{\meter}$ ab.
Die Rauigkeiten wurden zu $\sigma_{\text{Luft, Poly}} = \SI{2.9e-10}{}$ und $\sigma_{\text{Poly, SI}} = \SI{1.4e-11}{}$ bestimmt.
Der Parrattalgorithmus zeigt aufgrund der hohen Abweichungen der Parameter von den Literaturwerten eine schlechte Übereinstimmung mit den Messdaten auf.
Dies kann mehrere Ursachen haben, wie zum Beispiel eine ungenaue Justierung des Detektors oder des Winkels des Röntgenstrahls.
Aufgrund der vorgenommenen Auswertungen der Justage ist dies allerdings unwahrscheinlich.
Es kann aber auch auf eine mit Staub und anderen Verunreinigungen behaftete Probe zurückzuführen sein, die die Messdaten verfälscht haben könnte.
Der Parrattalgorithmus ist ein sehr komplexes Verfahren, das eine Vielzahl von Parametern benötigt, die alle mit Unsicherheiten behaftet sind, was ebenfalls zu den Abweichungen führen kann. \newline
Abschließend lässt sich sagen, dass die Justierungen des Versuches gut verliefen und die Auswertungen der Justierungen genau waren.
Es konnte außerdem eine Schichtdicke der Probe bestimmt werden.
Die Auswertung des Parrattalgorithmus war allerdings weniger genau und wies hohe Abweichungen von den Literatur- und zuvor bestimmten Werten auf, weshalb die Auswertung des Parrattalgorithmus als weniger genau angesehen werden kann.