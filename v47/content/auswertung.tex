\section{Auswertung}
\label{sec:Auswertung}
To compute the isobaric heat capacity $C_p$, the molar mass of copper $M = 63.546\frac{\text{g}}{\text{mol}}$, the sample's mass $m = \text{g}$, the energy $E$ that was added to the system and the temperature difference $\Delta T$ that was achieved by adding this amount of energy are required. \\
To get the energy, the voltage $U$, the current $I$ and the time interval $\Delta t$ over which energy was added to the system were measured. To get the temperature $T$ of the system, the thermometre's resistance $R$ was measured, since the temperature can then be measured via
\begin{aquation}
  T\lbr R \rbr = 0.00134 R^2 + 2.296 R - 243.02 \tp
\end{aquation}
The resistance steps have been chosen carefully to ensure that $\Delta T$ is always $10\text{K}$. To get the isobaric heat capacity, the measured and computed values have to be plugged into the equation 
\begin{aquation}
  C_p &= \frac{M}{m} \frac{E}{\Delta T}\tp
\end{aquation}
Die dafür notwendigen Daten finden sich in 
\begin{tabular}{rrr}
\toprule
$T$ / K & $E$c / J & $C_p$ / (J/(mol*K)) \\
\midrule
-190.00 & 0.00 & 0.00 \\
-180.00 & 830.96 & 15.44 \\
-170.00 & 927.58 & 17.24 \\
-160.00 & 917.38 & 17.05 \\
-150.00 & 940.83 & 17.48 \\
-140.00 & 959.08 & 17.82 \\
-130.00 & 1026.15 & 19.07 \\
-120.00 & 1062.44 & 19.74 \\
-110.00 & 1141.43 & 21.21 \\
-100.00 & 1152.03 & 21.41 \\
-90.00 & 1113.17 & 20.68 \\
-80.00 & 1110.73 & 20.64 \\
-70.00 & 1084.14 & 20.14 \\
-60.00 & 1301.73 & 24.19 \\
-50.00 & 1711.84 & 31.81 \\
-40.00 & 1309.45 & 24.33 \\
-30.00 & 1331.94 & 24.75 \\
-20.00 & 1241.91 & 23.08 \\
-10.00 & 1186.71 & 22.05 \\
-0.00 & 1084.19 & 20.15 \\
\bottomrule
\end{tabular}



% \begin{figure}
%   \centering
%   \includegraphics{plot.pdf}
%   \caption{Plot.}
%   \label{fig:plot}
% \end{figure}


Siehe \autoref{fig:plot} und \autoref{tab:tabelle}!
