\section{Diskussion}
\label{sec:Diskussion}
Der Literaturwert für die mittlere Lebensdauer von Müonen ist $\tau = 2.19703 \pm 0.00004 \text{$\mu$s}$. \cite{myon_chemie_de} Der experimentell bestimmte Wert von $\tau = (2.07 \pm 0.09)\text{$\mu$s}$ liegt nur knapp außerhalb der Standardabweichung und hat eine relative Abweichung von $5.8\%$. Grund dafür könnte sein, dass bei der Durchführung Bauteile beschädigt wurden, was dazu führte, dass eine höhere Betriebsspannung verwendet wurde, was die Spannungsspitze am Anfang des Plots verursachte. Da bei Exponentialfunktionen jedoch gerade dieser Bereich am stärksten zum Fit beiträgt, wurde die Varianz der Daten durch dessen Wegschneiden erhöht.

\section{Originaldaten}

\begin{longtable}{rrr}
\caption{Messwerte zur Eichung.}\\
\toprule
Pulsbreite & Kanal \\
\midrule
\endfirsthead
\toprule
Pulsbreite & Kanal \\
\midrule
\endhead
\midrule
\multicolumn{3}{r}{Continued on next page} \\
\midrule
\endfoot
\bottomrule
\endlastfoot
0.3 & 7 \\
1.3 & 53 \\
2.3 & 99 \\
3.3 & 145 \\
4.3 & 192 \\
5.3 & 238 \\
6.3 & 284 \\
7.3 & 330 \\
8.3 & 376 \\
\end{longtable}


\begin{longtable}{rr}
\caption{Messwerte zur Bestimmung der Lebenszeit.}\\
\toprule
Index & Counts \\
\midrule
\endfirsthead
\toprule
Index & Counts \\
\midrule
\endhead
\midrule
\multicolumn{2}{r}{Continued on next page} \\
\midrule
\endfoot
\bottomrule
\endlastfoot
0 & 0 \\
1 & 0 \\
2 & 0 \\
3 & 2 \\
4 & 20 \\
5 & 23 \\
6 & 17 \\
7 & 32 \\
8 & 24 \\
9 & 22 \\
10 & 21 \\
11 & 21 \\
12 & 27 \\
13 & 15 \\
14 & 26 \\
15 & 26 \\
16 & 29 \\
17 & 49 \\
18 & 130 \\
19 & 232 \\
20 & 258 \\
21 & 318 \\
22 & 273 \\
23 & 181 \\
24 & 139 \\
25 & 106 \\
26 & 62 \\
27 & 30 \\
28 & 35 \\
29 & 27 \\
30 & 25 \\
31 & 24 \\
32 & 17 \\
33 & 25 \\
34 & 15 \\
35 & 17 \\
36 & 22 \\
37 & 18 \\
38 & 19 \\
39 & 21 \\
40 & 16 \\
41 & 20 \\
42 & 16 \\
43 & 18 \\
44 & 17 \\
45 & 8 \\
46 & 19 \\
47 & 19 \\
48 & 18 \\
49 & 15 \\
50 & 15 \\
51 & 19 \\
52 & 18 \\
53 & 16 \\
54 & 16 \\
55 & 12 \\
56 & 10 \\
57 & 13 \\
58 & 15 \\
59 & 12 \\
60 & 15 \\
61 & 17 \\
62 & 15 \\
63 & 14 \\
64 & 13 \\
65 & 22 \\
66 & 9 \\
67 & 16 \\
68 & 12 \\
69 & 7 \\
70 & 12 \\
71 & 22 \\
72 & 15 \\
73 & 12 \\
74 & 11 \\
75 & 16 \\
76 & 11 \\
77 & 13 \\
78 & 15 \\
79 & 10 \\
80 & 21 \\
81 & 9 \\
82 & 10 \\
83 & 15 \\
84 & 8 \\
85 & 10 \\
86 & 7 \\
87 & 22 \\
88 & 4 \\
89 & 16 \\
90 & 10 \\
91 & 7 \\
92 & 7 \\
93 & 11 \\
94 & 9 \\
95 & 8 \\
96 & 5 \\
97 & 5 \\
98 & 9 \\
99 & 2 \\
100 & 4 \\
101 & 13 \\
102 & 7 \\
103 & 8 \\
104 & 7 \\
105 & 13 \\
106 & 14 \\
107 & 5 \\
108 & 5 \\
109 & 5 \\
110 & 9 \\
111 & 8 \\
112 & 12 \\
113 & 11 \\
114 & 11 \\
115 & 9 \\
116 & 6 \\
117 & 3 \\
118 & 9 \\
119 & 9 \\
120 & 6 \\
121 & 9 \\
122 & 7 \\
123 & 10 \\
124 & 9 \\
125 & 6 \\
126 & 9 \\
127 & 9 \\
128 & 4 \\
129 & 5 \\
130 & 9 \\
131 & 11 \\
132 & 7 \\
133 & 9 \\
134 & 8 \\
135 & 5 \\
136 & 7 \\
137 & 8 \\
138 & 8 \\
139 & 8 \\
140 & 9 \\
141 & 10 \\
142 & 3 \\
143 & 4 \\
144 & 5 \\
145 & 3 \\
146 & 3 \\
147 & 4 \\
148 & 12 \\
149 & 3 \\
150 & 3 \\
151 & 4 \\
152 & 7 \\
153 & 8 \\
154 & 2 \\
155 & 5 \\
156 & 12 \\
157 & 1 \\
158 & 7 \\
159 & 3 \\
160 & 3 \\
161 & 3 \\
162 & 7 \\
163 & 4 \\
164 & 6 \\
165 & 8 \\
166 & 2 \\
167 & 4 \\
168 & 3 \\
169 & 6 \\
170 & 3 \\
171 & 10 \\
172 & 3 \\
173 & 4 \\
174 & 3 \\
175 & 4 \\
176 & 5 \\
177 & 4 \\
178 & 3 \\
179 & 4 \\
180 & 7 \\
181 & 3 \\
182 & 7 \\
183 & 6 \\
184 & 4 \\
185 & 5 \\
186 & 4 \\
187 & 6 \\
188 & 2 \\
189 & 8 \\
190 & 9 \\
191 & 5 \\
192 & 6 \\
193 & 5 \\
194 & 3 \\
195 & 3 \\
196 & 0 \\
197 & 2 \\
198 & 4 \\
199 & 4 \\
200 & 3 \\
201 & 5 \\
202 & 5 \\
203 & 2 \\
204 & 3 \\
205 & 3 \\
206 & 3 \\
207 & 4 \\
208 & 4 \\
209 & 4 \\
210 & 2 \\
211 & 5 \\
212 & 1 \\
213 & 1 \\
214 & 3 \\
215 & 2 \\
216 & 4 \\
217 & 6 \\
218 & 2 \\
219 & 3 \\
220 & 2 \\
221 & 5 \\
222 & 5 \\
223 & 3 \\
224 & 3 \\
225 & 2 \\
226 & 1 \\
227 & 3 \\
228 & 4 \\
229 & 5 \\
230 & 2 \\
231 & 2 \\
232 & 4 \\
233 & 3 \\
234 & 0 \\
235 & 2 \\
236 & 3 \\
237 & 3 \\
238 & 3 \\
239 & 3 \\
240 & 4 \\
241 & 1 \\
242 & 3 \\
243 & 2 \\
244 & 3 \\
245 & 3 \\
246 & 0 \\
247 & 1 \\
248 & 1 \\
249 & 1 \\
250 & 1 \\
251 & 4 \\
252 & 1 \\
253 & 3 \\
254 & 1 \\
255 & 3 \\
256 & 2 \\
257 & 5 \\
258 & 2 \\
259 & 0 \\
260 & 0 \\
261 & 2 \\
262 & 2 \\
263 & 2 \\
264 & 1 \\
265 & 1 \\
266 & 3 \\
267 & 0 \\
268 & 2 \\
269 & 1 \\
270 & 6 \\
271 & 2 \\
272 & 1 \\
273 & 1 \\
274 & 0 \\
275 & 3 \\
276 & 0 \\
277 & 0 \\
278 & 1 \\
279 & 2 \\
280 & 0 \\
281 & 2 \\
282 & 2 \\
283 & 2 \\
284 & 1 \\
285 & 3 \\
286 & 1 \\
287 & 2 \\
288 & 2 \\
289 & 0 \\
290 & 1 \\
291 & 1 \\
292 & 0 \\
293 & 1 \\
294 & 0 \\
295 & 2 \\
296 & 1 \\
297 & 1 \\
298 & 2 \\
299 & 1 \\
300 & 0 \\
301 & 1 \\
302 & 1 \\
303 & 0 \\
304 & 1 \\
305 & 1 \\
306 & 3 \\
307 & 1 \\
308 & 0 \\
309 & 1 \\
310 & 1 \\
311 & 3 \\
312 & 0 \\
313 & 3 \\
314 & 4 \\
315 & 0 \\
316 & 3 \\
317 & 1 \\
318 & 3 \\
319 & 0 \\
320 & 1 \\
321 & 0 \\
322 & 2 \\
323 & 0 \\
324 & 1 \\
325 & 1 \\
326 & 1 \\
327 & 0 \\
328 & 2 \\
329 & 2 \\
330 & 2 \\
331 & 1 \\
332 & 0 \\
333 & 3 \\
334 & 0 \\
335 & 1 \\
336 & 1 \\
337 & 0 \\
338 & 0 \\
339 & 0 \\
340 & 2 \\
341 & 0 \\
342 & 1 \\
343 & 0 \\
344 & 0 \\
345 & 0 \\
346 & 0 \\
347 & 1 \\
348 & 2 \\
349 & 3 \\
350 & 1 \\
351 & 0 \\
352 & 1 \\
353 & 0 \\
354 & 1 \\
355 & 1 \\
356 & 2 \\
357 & 1 \\
358 & 1 \\
359 & 0 \\
360 & 2 \\
361 & 0 \\
362 & 3 \\
363 & 2 \\
364 & 0 \\
365 & 0 \\
366 & 2 \\
367 & 0 \\
368 & 1 \\
369 & 1 \\
370 & 1 \\
371 & 1 \\
372 & 0 \\
373 & 0 \\
374 & 1 \\
375 & 0 \\
376 & 0 \\
377 & 2 \\
378 & 0 \\
379 & 0 \\
380 & 0 \\
381 & 1 \\
382 & 0 \\
383 & 1 \\
384 & 0 \\
385 & 1 \\
386 & 1 \\
387 & 1 \\
388 & 2 \\
389 & 0 \\
390 & 1 \\
391 & 4 \\
392 & 1 \\
393 & 0 \\
394 & 1 \\
395 & 1 \\
396 & 0 \\
397 & 2 \\
398 & 1 \\
399 & 0 \\
400 & 0 \\
401 & 1 \\
402 & 0 \\
403 & 1 \\
404 & 0 \\
405 & 0 \\
406 & 1 \\
407 & 0 \\
408 & 1 \\
409 & 1 \\
410 & 0 \\
411 & 0 \\
412 & 1 \\
413 & 0 \\
414 & 0 \\
415 & 0 \\
416 & 0 \\
417 & 0 \\
418 & 0 \\
419 & 0 \\
420 & 0 \\
421 & 0 \\
422 & 0 \\
423 & 0 \\
424 & 0 \\
425 & 0 \\
426 & 0 \\
427 & 0 \\
428 & 0 \\
429 & 0 \\
430 & 0 \\
431 & 0 \\
432 & 0 \\
433 & 0 \\
434 & 0 \\
435 & 0 \\
436 & 0 \\
437 & 0 \\
438 & 0 \\
439 & 0 \\
440 & 0 \\
441 & 0 \\
442 & 0 \\
443 & 0 \\
444 & 0 \\
445 & 0 \\
446 & 0 \\
447 & 0 \\
448 & 0 \\
449 & 0 \\
450 & 0 \\
451 & 0 \\
452 & 0 \\
453 & 0 \\
454 & 0 \\
455 & 0 \\
456 & 0 \\
457 & 0 \\
458 & 0 \\
459 & 0 \\
460 & 0 \\
461 & 0 \\
462 & 0 \\
463 & 0 \\
464 & 0 \\
465 & 0 \\
466 & 0 \\
467 & 0 \\
468 & 0 \\
469 & 0 \\
470 & 0 \\
471 & 0 \\
472 & 0 \\
473 & 0 \\
474 & 0 \\
475 & 0 \\
476 & 0 \\
477 & 0 \\
478 & 0 \\
479 & 0 \\
480 & 0 \\
481 & 0 \\
482 & 0 \\
483 & 0 \\
484 & 0 \\
485 & 0 \\
486 & 0 \\
487 & 0 \\
488 & 0 \\
489 & 0 \\
490 & 0 \\
491 & 0 \\
492 & 0 \\
493 & 0 \\
494 & 0 \\
495 & 0 \\
496 & 0 \\
497 & 0 \\
498 & 0 \\
499 & 0 \\
500 & 0 \\
501 & 0 \\
502 & 0 \\
503 & 0 \\
504 & 0 \\
505 & 0 \\
506 & 0 \\
507 & 0 \\
508 & 0 \\
509 & 0 \\
510 & 0 \\
\end{longtable}
