\documentclass[
  bibliography=totoc,     % Literatur im Inhaltsverzeichnis
  captions=tableheading,  % Tabellenüberschriften
  titlepage=firstiscover, % Titelseite ist Deckblatt
]{scrartcl}

% Paket float verbessern
\usepackage{scrhack}

% Warnung, falls nochmal kompiliert werden muss
\usepackage[aux]{rerunfilecheck}

% unverzichtbare Mathe-Befehle
\usepackage{amsmath}
% viele Mathe-Symbole
\usepackage{amssymb}
% Erweiterungen für amsmath
\usepackage{mathtools}

% Fonteinstellungen
\usepackage{fontspec}
% Latin Modern Fonts werden automatisch geladen
% Alternativ zum Beispiel:
%\setromanfont{Libertinus Serif}
%\setsansfont{Libertinus Sans}
%\setmonofont{Libertinus Mono}

% Wenn man andere Schriftarten gesetzt hat,
% sollte man das Seiten-Layout neu berechnen lassen
\recalctypearea{}

% deutsche Spracheinstellungen
\usepackage[ngerman]{babel}


\usepackage[
  math-style=ISO,    % ┐
  bold-style=ISO,    % │
  sans-style=italic, % │ ISO-Standard folgen
  nabla=upright,     % │
  partial=upright,   % ┘
  warnings-off={           % ┐
    mathtools-colon,       % │ unnötige Warnungen ausschalten
    mathtools-overbracket, % │
  },                       % ┘
]{unicode-math}

% traditionelle Fonts für Mathematik
\setmathfont{Latin Modern Math}
% Alternativ zum Beispiel:
%\setmathfont{Libertinus Math}

\setmathfont{XITS Math}[range={scr, bfscr}]
\setmathfont{XITS Math}[range={cal, bfcal}, StylisticSet=1]

% Zahlen und Einheiten
\usepackage[
  locale=DE,                   % deutsche Einstellungen
  separate-uncertainty=true,   % immer Unsicherheit mit \pm
  per-mode=symbol-or-fraction, % / in inline math, fraction in display math
]{siunitx}

% chemische Formeln
\usepackage[
  version=4,
  math-greek=default, % ┐ mit unicode-math zusammenarbeiten
  text-greek=default, % ┘
]{mhchem}

% richtige Anführungszeichen
\usepackage[autostyle]{csquotes}

% schöne Brüche im Text
\usepackage{xfrac}

% Standardplatzierung für Floats einstellen
\usepackage{float}
\floatplacement{figure}{htbp}
\floatplacement{table}{htbp}

% Floats innerhalb einer Section halten
\usepackage[
  section, % Floats innerhalb der Section halten
  below,   % unterhalb der Section aber auf der selben Seite ist ok
]{placeins}

% Seite drehen für breite Tabellen: landscape Umgebung
\usepackage{pdflscape}

% Captions schöner machen.
\usepackage[
  labelfont=bf,        % Tabelle x: Abbildung y: ist jetzt fett
  font=small,          % Schrift etwas kleiner als Dokument
  width=0.9\textwidth, % maximale Breite einer Caption schmaler
]{caption}
% subfigure, subtable, subref
\usepackage{subcaption}

% Grafiken können eingebunden werden
\usepackage{graphicx}

% schöne Tabellen
\usepackage{booktabs}

% Verbesserungen am Schriftbild
\usepackage{microtype}

% Literaturverzeichnis
\usepackage[
  backend=biber,
  sorting=none,
]{biblatex}
% Quellendatenbank
\addbibresource{lit.bib}
\addbibresource{programme.bib}

% Hyperlinks im Dokument
\usepackage[
  german,
  unicode,        % Unicode in PDF-Attributen erlauben
  pdfusetitle,    % Titel, Autoren und Datum als PDF-Attribute
  pdfcreator={},  % ┐ PDF-Attribute säubern
  pdfproducer={}, % ┘
]{hyperref}
% erweiterte Bookmarks im PDF
\usepackage{bookmark}

% Trennung von Wörtern mit Strichen
\usepackage[shortcuts]{extdash}

% Listings zum Einbinden von Quellcode
\usepackage{listings}

% BraKet-Notation
\usepackage{braket}

% Feynman Slashed
\usepackage{slashed}

% New Commands Axel
% aquation = aligned equations
\newenvironment{aquation}{\begin{equation}\begin{aligned}}{\end{aligned}\end{equation}}
% \text{ .}/, 
\newcommand{\tp}{\text{ . }}
\newcommand{\tc}{\text{ , }}
% shorthand for 1/2 
\newcommand{\oh}{\frac{1}{2}}
\newcommand{\ih}{\frac{i}{2}}
% shorthand for f(x)=1/x 
\newcommand{\odiv}[1]{\frac{1}{#1}}
\newcommand{\idiv}[1]{\frac{i}{#1}}

%%%%% shorthands for left/right brackets (r for round, s for square, c for curly)
% round brackets
\newcommand{\lbr}{\left(}
\newcommand{\rbr}{\right)}
\newcommand{\brr}[1]{\left( #1 \right)}
% square brackets 
\newcommand{\lbs}{\left[}
\newcommand{\rbs}{\right]}
\newcommand{\brs}[1]{\left[ #1 \right]}
% curly brackets 
\newcommand{\lbc}{\left{}
\newcommand{\rbc}{\right}}
\newcommand{\brc}[1]{\left{ #1 \right}}
% put stuff like \\ , \\& , etc. into brackets (by putting it outside)
\newcommand{\brbr}[1]{\right. #1 \left.}

%%%%% mathcal L, O (for Lagrangian densities and and SMEFT operators)
\newcommand{\Lm}{\mathcal{L}}
\newcommand{\Om}{\mathcal{O}}

%%%%% horizintal .5mm
\newcommand{\hsf}{\hspace{0.5mm}}

% kiloOhm, kiloHertz
\newcommand{\kOhm}{\text{k$\Omega$}}
\newcommand{\kHz}{\text{kHz}}

\author{%
  Joel Koch\\%
  \href{mailto:joel.koch@tu-dortmund.de}{joel.koch@tu-dortmund.de}
  \and%
  % Felix Symma\\%
  % \href{mailto:felix.symma@tu-dortmund.de}{felix.symma@tu-dortmund.de}%
  Axel Vogt \\%
  \href{mailto:axel.vogt@udo.edu}{axel.vogt@udo.edu}
}
\publishers{TU Dortmund – Fakultät Physik}


\subject{v44}
\title{Röntgenreflektometrie}
\date{%
  Durchführung: 28.10.2024
  \hspace{3em}
  Abgabe: 29.11.2024
}

\begin{document}

\maketitle
\thispagestyle{empty}
\tableofcontents
\newpage

\DeclareSIUnit\angstrom{Å}

\section{Zielsetzung}
\label{sec:zielsetzung}
Das vorliegende Experiment untersucht Festkörper, indem diese mit Röntgenstrahlung beschossen werden. Auf diese Weise können verschiedene Materialparameter gemessen werden. Im vorliegenden Fall wurden Schichtdicke, Rauigkeit und Dispersion eines dünnen Films, welcher auf einen Siliziumwafer aufgebracht wurden, untersucht.
% \pagebreak
\section{Zielsetzung}\label{sec:Zielsetzung}

Ziel dieses Versuches ist es die Lebensdauer kosmischer Myonen zu bestimmen, um dabei grundlegende Detektorkomponenten näher kennen zu lernen.

\section{Theorie}\label{sec:Theorie}

Die Informationen über die Teilchen werden der particle data group entnommen\cite{ParticleDataGroup:2024cfk}. \newline
Myonen sind Elementarteilchen aus der zweiten Generation der Fermionen und gehören zur Leptonenfamilie.
Die Myonen, die auf der Erde gemessen werden können, sind Sekundärteilchen, die bei der Wechselwirkung von hochenergetischer kosmischer Strahlung von zum Beispiel der Sonne mit den oberen Schichten der Erdatmosphäre entstehen.
Sie entstehen in einer Höhe von etwa $\SI{15}{\kilo\meter}$.
Trifft ein Proton auf ein Atom der Erdatmosphäre, entstehen bei der Kollision Sekundärteilchen.
Da Pionen und Kaonen die leichtesten Mesonen sind, entstehen diese bei der Kollision am häufigsten.
Diese wiederum zerfallen über die elektromagnetische Wechselwirkung in Myonen und Myon-Neutrinos via der folgenen Zerfallskanäle
\begin{align*}
    \pi^\pm&\rightarrow \mu^\pm\nu_\mu(\bar\nu_\mu),\\
    K^\pm&\rightarrow\mu^\pm\nu_\mu(\bar\nu_\mu).
\end{align*}
Um die Impulserhaltung einzuhalten, wird neben einem Myonen auch ein Myon-Neutrino $\nu_\mu$ (bei Zerfall eines $\pi^+$) oder Antineutrino $\bar\nu_\mu$ erzeugt (bei Zerfall eines $\pi^-$).
Selbiges gilt für die Kaonen. 
Die Myonen selber zerfallen in Elektronen und Elektron-Antineutrinos via
\begin{align*}
    \mu^\pm &\rightarrow e^\pm \nu_e(\bar\nu_e) \bar\nu_\mu (\nu_\mu).
\end{align*}
\newline
Die durchschnittliche Zeit, die ein Teilchen benötigt, um in andere Teilchen zu zerfallen wird (mittlere) Zerfallszeit $\tau$ genannt.
Sie folgt über das Zerfallsgesetz 
\begin{align}
    N(t)=N_0e^{-\lambda t},
\end{align}
wobei $N(t)$ die Anzahl der noch nicht zerfallenen Teilchen zum Zeitpunkt $t$ ist und $N_0$ die Anzahl der Teilchen zum Zeitpunkt $t=0$.
Die Zerfallskonstante $\lambda$ ist dabei ein Proportionalitätsfaktor, der für jedes Teilchen unterscheidlich ist.
Mithilfe des Zerfallsgesetzes kann nun die mittlere Lebensdauer $\tau$ eines Teilchens durch zeitliches Ableiten und umstellen bestimmt werden,
\begin{align}
    \frac{N(t)}{N_0}&=\lambda e^{-\lambda t}\text dt, \\
    \tau = \braket t &= \int_0^\infty t\lambda e^{-\lambda t}\text dt = \frac{1}{\lambda}.
\end{align}
Dass die Myonen auf der Erde auftreffenn lässt sich nach dem klassischen Modell der Physik nicht beschreiben, da die Lebensdauer der Myonen von $\SI{2.197}{\micro\second}$ zu kurz wäre, um die Erde zu erreichen,
\begin{align*}
    s = v\cdot t\approx \SI{3e8}{\meter\per\second}\cdot \SI{2.196e-6}{\second}\approx \SI{660}{\meter}.
\end{align*}
Die Myonen erreichen die Erde jedoch, da sie aufgrund ihrer hohen Geschwindigkeit eine Zeitdilatation erfahren, die nach der speziellen Relativitätstheorie von Albert Einstein beschrieben wird,
\begin{align*}
    \gamma = \frac{E}{m_\mu c^2} = \frac{10\cdot 10^9}{105,66\cdot 10^6}\approx94.7, \\
    \rightarrow s = vt\cdot\gamma\approx \SI{3e8}{\meter\per\second}\cdot \SI{2.196e-6}{\second} \cdot 94.7\approx \SI{62.4}{\kilo\meter}.
\end{align*}
Somit bestimmt sich die Wahrscheinlichkeit für ein Myon auf der Erde aufzutreffen zu
\begin{align*}
    P&=e^{-\sfrac{L}{s}}\ \text{mit}\ L=\SI{15}{\kilo\meter}, \\
    P_{\text{klassisch}} &= e^{-\sfrac{L}{\SI{660}{\meter}}}\approx 1.3\cdot 10^{-10}, \\
    P_{\text{relativistisch}} &= e^{-\sfrac{L}{\SI{62.4}{\kilo\meter}}}\approx 78.63\%.
\end{align*}
% \pagebreak
\section{Durchführung}
\label{sec:Durchführung}
Das hier verwendete Diffraktometer besteht aus einer Röntgenröhre auf der einen und einem Empfänger für Röntgenstrahlung auf der anderen Seite. Mittig zwischen ihnen ist ein Tisch für die Probe platziert. Zur Steuerung des Versuchsaufbau wird das Programm XRD-Commander verwendet.

\subsection{Die Röntgenröhre}
\label{subsec:Röntgenröhre}
In diesem Versuch wird eine Röntgenröhre mit Cu-Anode verwendet. In dieser werden Elektronen aus einer Glühkathode in Richtung Cu-Anode beschleunigt. Die in die Cu-Anode einschlagenden Elektronen interagieren dann mit dem Kupfer, und zwar auf zweierlei Arten: Durch Stöße mit den Elektronen in den Kupferatomen werden diese auf höhere Niveaus angeregt. meistens werden die Atome sogar ionisiert. Dies macht Plätze auf niedrigeren Niveaus frei und führt damit dazu, dass Elektronen höherer Niveaus unter Emission von Röntgenstrahlung auf auf diese niedrigeren Niveaus zurückfallen. Außerdem wird bei der Bewegung der Elektronen im elektrischen Feld innerhalb des Kupfers Bremsstrahlung emittiert.\\
In diesem Versuch wird die $\text{K}_\alpha$-Linie von Kupfer verwendet, also jene Röntgenstrahlung, welche beim Niveauüübergang von der zweiten auf die erste Schale entsteht. Sie besitzt eine Wellenlänge von $\lambda=1.54\text{Å}$. Um diese aus dem restlichen Spektrum der Röntgenröhre herauszufiltern, wird ein \textit{Göbelspiegel} verwendet. Dieser bündelt die Strahlung und filtert alle anderen Linien aus der Strahlung heraus.\\
Die Strahlung fällt nach dem Göbelspiegel auf eine Blende, dann auf die Probe und dann auf eine weitere Blende. Die Blenden dienen dazu, den eingestellten Glanzwinkel der einfallenden und den gemessenen Glanzwinkel der ausfallenden Strahlung zu variieren.\\
\\
Bei sehr geringen Winkeln (kleiner als $\alpha_g$) schießt die Strahlung über die Probe hinaus. Dies äußert sich in einem reskalierenden Faktor vor der gemessenen reflektierten Intensität, dem \textit{Geometriefaktor} 
\begin{aquation}
    G &\coloneqq \left\{ \begin{matrix}
        \frac{D \sin{\alpha_i}}{d_0} & \alpha_i < \alpha_g \\
        1 & \text{sonst}
    \end{matrix} \right. \tc
    \label{eq:Geometriefaktor}
\end{aquation}
mit der Probenlänge $D$ und der Strahlbreite $d_0$.

\subsection{Justage}
Da die Probe, welche hier ein mit einer dünnen Polystyrolschicht überzogener Silizium-Wafer ist, nach dem Einbringen in den Versuchsaufbau noch nicht die richtige Ausrichtung hat, muss der Aufbau zunächst justiert werden. Ziel der Justage ist es, die Probe genau mittig zwischen Sender und Empfänger zu platzieren, sodass die gesamte Strahlung der Röntgenröhre auf die Probe trifft. Außerdem muss die Nullposition für Sender und Empfänger so justiert werden, dass die Probe bei einem Glanzwinkel von null genau parallel zum Strahl steht.\\
\\
Im ersten Schritt der Justage wird daher ein Detektorscan durchgeführt. Bei diesem wird die Röntgenröhre direkt auf den Detektor gerichtet. Der Detektor wird im Winkel variiert und durch den Strahl gefahren. Das Maximum dieser Gauß-verteilten Messung wird dann als neue Null-Position eingestellt.\\
\\
Als Nächstes wird die Probe in den Strahl eingebracht. Da zuvor die der gesamte Strahl den Detektor traf, lässt sich durch Hochfahren Bewegen der Probe im Strahl der Punkt in z-Richtung bestimmen, bei welchem ungefähr der halbe Strahl abgeschattet ist. Dies wird mithilfe eines \textit{Z-Scan}s bewerkstelligt. Auf diesen Punkt wird die Probe gesetzt. Da die Feinjustage später kommt, reicht hier ein ungefährer Wert.\\
\\
Der folgende Scan dient dazu, den Einfallswinkel gleich dessen Ausfallswinkel zu setzen. Dazu werden Röntgenröhre und Detektor um die Probe rotiert, ohne ihre relative Ausrichtung zu einander zu ändern. Dieser Scan heißt \textit{Rocking-Scan} (engl. \textit{to rock}: Schaukeln). Da dieser Scan den Strahl einmal mit der dem Emitter und einmal mit der dem Detektor zugewandten Kante der Probe verdeckt, liefert er Aufschluss über die Positionierung der Probe entlang des Strahls. Das Ergebnis dieses Scans sollte ein gleichseitiges Dreieck sein. Ist dies nicht der Fall, muss die Probe entlang des Strahls verschoben werden, bis ihre Ausrichtung passt, damit später auch wirklich der gesamte Strahl die Probe trifft. Nach dieser Justage wird das Maximum des Rockingscans als neue Nulllage des Emitter-Detektor-Paars gewählt, damit der Strahl nun genau parallel zur Oberfläche der Probe steht.\\
\\
Im Anschluss wird ein weiterer Z-Scan durchgeführt, um eine präzisere halbe Abschattung des Strahls zu erreichen.\\
\\
Ein anschließender Rocking-Scan, mit einem Glanzwinkel von $0.15^\circ$ verfeinert die Parallelität der Probe zum Strahl.\\
\\
Zum Schluss wird für den selben Glanzwinkel ein letzter Z-Scan durchgeführt und die Probe wieder auf die halbe Abschattung gesetzt. Dieser sorgt dafür, dass der Strahl genau mittig auf die Probe trifft.

\subsection{Reflektivitätsmessung}
Die eigentliche Messung dient zur Bestimmung der Reflektivität. Um diese zu bestimmen, muss wie zuvor beschrieben ein Winkelbereich abgefahren werden. Dabei werden Emitter und Detektor gleichzeitig immer um den selben Glanzwinkel verschoben, um so die Intensität des reflektierten Strahls unter verschiedenen Winkel zu messen. Hier wurde ein Winkelintervall von $[0^\circ,2.5^\circ]$ gewählt, mit einer Schrittweite von $0.005^\circ$.\\
\\
Um den Untergrund der Reflektivitätsmessung zu bestimmen, wird der selbe Scan erneut durchgeführt, allerdings mit um $0.1^\circ$ verschobenem Detektor, was als \textit{diffuser Scan} bezeichnet wird. 
\cite{Anleitung44}
% \pagebreak
\section{Analysis}
\label{sec:Analysis}

\subsection{Dependence of the contrast on the polarisation angle}
\label{subsec:polarisation}


\begin{table}[H]
    \centering
    \caption{Measured Values of the contrast in dependence on the polarisation angle.}
    \label{tab:values_polarisation}
    \begin{tabular}{c c c c}
        \toprule
        $\vartheta \,/\, \si{\degree}$ & $I_{\text{max}}\,/\,\si{\volt}$ & $I_{\text{min}}\,/\,\si{\volt}$ & contrast \\
        \midrule
        0.0 & $5\pm4$ & $4.5\pm1.5$ & $0.0\pm0.4$ \\
        15.0 & $4\pm4$ & $2.4\pm1.5$ & $0.3\pm0.5$ \\
        30.0 & $4\pm4$ & $1.1\pm1.5$ & $0.5\pm0.6$ \\
        45.0 & $4\pm4$ & $0.8\pm1.5$ & $0.7\pm0.5$ \\
        60.0 & $6\pm4$ & $1.1\pm1.5$ & $0.7\pm0.4$ \\
        75.0 & $7\pm4$ & $2.3\pm1.5$ & $0.48\pm0.35$ \\
        90.0 & $7\pm4$ & $5.4\pm1.5$ & $0.14\pm0.31$ \\
        105.0 & $10\pm4$ & $4.4\pm1.5$ & $0.38\pm0.23$ \\
        120.0 & $13\pm4$ & $2.7\pm1.5$ & $0.66\pm0.18$ \\
        135.0 & $18\pm4$ & $1.7\pm1.5$ & $0.83\pm0.15$ \\
        150.0 & $13\pm4$ & $2.5\pm1.5$ & $0.67\pm0.18$ \\
        165.0 & $10\pm4$ & $3.9\pm1.5$ & $0.44\pm0.23$ \\
        180.0 & $5\pm4$ & $4.8\pm1.5$ & $0.1\pm0.4$ \\
        \bottomrule
    \end{tabular}
  \end{table}


  \begin{figure}[H]
    \centering
    \includegraphics[width=0.75\textwidth]{build/contrast.pdf}
    \caption{Graphical representation of the contrast in dependence on the polarisation angle.}
    \label{fig:contrast}
  \end{figure}

\subsection{Refraction index of glass}
\label{subsec:refraction_glass}

\begin{table}[htbp]
    \centering
    \begin{tabular}{c c c}
        \toprule
        Measurement series & $n$ \\
        \midrule
        1 & 31 \\
        2 & 34 \\
        3 & 34 \\
        4 & 33 \\
        5 & 35 \\
        \bottomrule
    \end{tabular}
    \caption{Measured counts for the refraction index of glass for different measurement series with $\upDelta \vartheta=\SI{10}{\degree}$.}
    \label{tab:refraction_glass}
\end{table}

\subsection{Refraction index of air}
\label{subsec:refraction_air}

\begin{table}[htbp]
    \centering
    \begin{tabular}{c c}
        \toprule
        pressure / $\si{\milli\bar}$  & counts \\    
        \midrule
        50.0 & $2.0$\\
        100.0 & $4.0$\\
        150.0 & $6.0$\\
        200.0 & $8.0$\\
        250.0 & $10.0$\\
        300.0 & $12.0$\\
        350.0 & $14.0$\\
        400.0 & $16.0$\\
        450.0 & $18.5\pm0.6$\\
        500.0 & $21.0$\\
        550.0 & $23.0$\\
        600.0 & $25.0$\\
        650.0 & $27.0$\\
        700.0 & $29.0$\\
        750.0 & $31.0\pm0.6$\\
        800.0 & $33.0\pm0.6$\\
        850.0 & $36.0$\\
        900.0 & $38.0$\\
        950.0 & $40.0$\\
        1000.0 & $42.0$\\
    \end{tabular}
    \caption{Measured counts for the refraction index of air in dependence of the pressure.}
    \label{tab:refraction_air}
\end{table}

\begin{figure}[H]
    \centering
    \includegraphics[width=0.75\textwidth]{build/refraction_index.pdf}
    \caption{Graphical representation of the refraction index of air in dependence of the pressure.}
    \label{fig:refraction_index}
  \end{figure}
% \pagebreak
\section{Diskussion}
\label{sec:Diskussion}
Der Literaturwert für die mittlere Lebensdauer von Myonen ist $\tau = 2,19703 \pm 0,00004 \text{$\mu$s}$. \cite{myon_chemie_de} Der experimentell bestimmte Wert von $\tau = (2,07 \pm 0,09)\text{$\mu$s}$ liegt nur knapp außerhalb der Standardabweichung und hat eine relative Abweichung von $5.8\%$. Grund dafür könnte sein, dass bei der Durchführung Bauteile beschädigt wurden, was dazu führte, dass eine höhere Betriebsspannung verwendet wurde, was die Spannungsspitze am Anfang des Plots verursachte. Da bei Exponentialfunktionen jedoch gerade dieser Bereich am stärksten zum Fit beiträgt, wurde die Varianz der Daten durch dessen Wegschneiden erhöht.

\section{Originaldaten}

\begin{longtable}{rrr}
\caption{Messwerte zur Eichung.}\\
\toprule
Pulsbreite & Kanal \\
\midrule
\endfirsthead
\toprule
Pulsbreite & Kanal \\
\midrule
\endhead
\midrule
\multicolumn{3}{r}{Continued on next page} \\
\midrule
\endfoot
\bottomrule
\endlastfoot
0.3 & 7 \\
1.3 & 53 \\
2.3 & 99 \\
3.3 & 145 \\
4.3 & 192 \\
5.3 & 238 \\
6.3 & 284 \\
7.3 & 330 \\
8.3 & 376 \\
\end{longtable}


\begin{longtable}{rr}
\caption{Messwerte zur Bestimmung der Lebenszeit.}\\
\toprule
Index & Counts \\
\midrule
\endfirsthead
\toprule
Index & Counts \\
\midrule
\endhead
\midrule
\multicolumn{2}{r}{Continued on next page} \\
\midrule
\endfoot
\bottomrule
\endlastfoot
0 & 0 \\
1 & 0 \\
2 & 0 \\
3 & 2 \\
4 & 20 \\
5 & 23 \\
6 & 17 \\
7 & 32 \\
8 & 24 \\
9 & 22 \\
10 & 21 \\
11 & 21 \\
12 & 27 \\
13 & 15 \\
14 & 26 \\
15 & 26 \\
16 & 29 \\
17 & 49 \\
18 & 130 \\
19 & 232 \\
20 & 258 \\
21 & 318 \\
22 & 273 \\
23 & 181 \\
24 & 139 \\
25 & 106 \\
26 & 62 \\
27 & 30 \\
28 & 35 \\
29 & 27 \\
30 & 25 \\
31 & 24 \\
32 & 17 \\
33 & 25 \\
34 & 15 \\
35 & 17 \\
36 & 22 \\
37 & 18 \\
38 & 19 \\
39 & 21 \\
40 & 16 \\
41 & 20 \\
42 & 16 \\
43 & 18 \\
44 & 17 \\
45 & 8 \\
46 & 19 \\
47 & 19 \\
48 & 18 \\
49 & 15 \\
50 & 15 \\
51 & 19 \\
52 & 18 \\
53 & 16 \\
54 & 16 \\
55 & 12 \\
56 & 10 \\
57 & 13 \\
58 & 15 \\
59 & 12 \\
60 & 15 \\
61 & 17 \\
62 & 15 \\
63 & 14 \\
64 & 13 \\
65 & 22 \\
66 & 9 \\
67 & 16 \\
68 & 12 \\
69 & 7 \\
70 & 12 \\
71 & 22 \\
72 & 15 \\
73 & 12 \\
74 & 11 \\
75 & 16 \\
76 & 11 \\
77 & 13 \\
78 & 15 \\
79 & 10 \\
80 & 21 \\
81 & 9 \\
82 & 10 \\
83 & 15 \\
84 & 8 \\
85 & 10 \\
86 & 7 \\
87 & 22 \\
88 & 4 \\
89 & 16 \\
90 & 10 \\
91 & 7 \\
92 & 7 \\
93 & 11 \\
94 & 9 \\
95 & 8 \\
96 & 5 \\
97 & 5 \\
98 & 9 \\
99 & 2 \\
100 & 4 \\
101 & 13 \\
102 & 7 \\
103 & 8 \\
104 & 7 \\
105 & 13 \\
106 & 14 \\
107 & 5 \\
108 & 5 \\
109 & 5 \\
110 & 9 \\
111 & 8 \\
112 & 12 \\
113 & 11 \\
114 & 11 \\
115 & 9 \\
116 & 6 \\
117 & 3 \\
118 & 9 \\
119 & 9 \\
120 & 6 \\
121 & 9 \\
122 & 7 \\
123 & 10 \\
124 & 9 \\
125 & 6 \\
126 & 9 \\
127 & 9 \\
128 & 4 \\
129 & 5 \\
130 & 9 \\
131 & 11 \\
132 & 7 \\
133 & 9 \\
134 & 8 \\
135 & 5 \\
136 & 7 \\
137 & 8 \\
138 & 8 \\
139 & 8 \\
140 & 9 \\
141 & 10 \\
142 & 3 \\
143 & 4 \\
144 & 5 \\
145 & 3 \\
146 & 3 \\
147 & 4 \\
148 & 12 \\
149 & 3 \\
150 & 3 \\
151 & 4 \\
152 & 7 \\
153 & 8 \\
154 & 2 \\
155 & 5 \\
156 & 12 \\
157 & 1 \\
158 & 7 \\
159 & 3 \\
160 & 3 \\
161 & 3 \\
162 & 7 \\
163 & 4 \\
164 & 6 \\
165 & 8 \\
166 & 2 \\
167 & 4 \\
168 & 3 \\
169 & 6 \\
170 & 3 \\
171 & 10 \\
172 & 3 \\
173 & 4 \\
174 & 3 \\
175 & 4 \\
176 & 5 \\
177 & 4 \\
178 & 3 \\
179 & 4 \\
180 & 7 \\
181 & 3 \\
182 & 7 \\
183 & 6 \\
184 & 4 \\
185 & 5 \\
186 & 4 \\
187 & 6 \\
188 & 2 \\
189 & 8 \\
190 & 9 \\
191 & 5 \\
192 & 6 \\
193 & 5 \\
194 & 3 \\
195 & 3 \\
196 & 0 \\
197 & 2 \\
198 & 4 \\
199 & 4 \\
200 & 3 \\
201 & 5 \\
202 & 5 \\
203 & 2 \\
204 & 3 \\
205 & 3 \\
206 & 3 \\
207 & 4 \\
208 & 4 \\
209 & 4 \\
210 & 2 \\
211 & 5 \\
212 & 1 \\
213 & 1 \\
214 & 3 \\
215 & 2 \\
216 & 4 \\
217 & 6 \\
218 & 2 \\
219 & 3 \\
220 & 2 \\
221 & 5 \\
222 & 5 \\
223 & 3 \\
224 & 3 \\
225 & 2 \\
226 & 1 \\
227 & 3 \\
228 & 4 \\
229 & 5 \\
230 & 2 \\
231 & 2 \\
232 & 4 \\
233 & 3 \\
234 & 0 \\
235 & 2 \\
236 & 3 \\
237 & 3 \\
238 & 3 \\
239 & 3 \\
240 & 4 \\
241 & 1 \\
242 & 3 \\
243 & 2 \\
244 & 3 \\
245 & 3 \\
246 & 0 \\
247 & 1 \\
248 & 1 \\
249 & 1 \\
250 & 1 \\
251 & 4 \\
252 & 1 \\
253 & 3 \\
254 & 1 \\
255 & 3 \\
256 & 2 \\
257 & 5 \\
258 & 2 \\
259 & 0 \\
260 & 0 \\
261 & 2 \\
262 & 2 \\
263 & 2 \\
264 & 1 \\
265 & 1 \\
266 & 3 \\
267 & 0 \\
268 & 2 \\
269 & 1 \\
270 & 6 \\
271 & 2 \\
272 & 1 \\
273 & 1 \\
274 & 0 \\
275 & 3 \\
276 & 0 \\
277 & 0 \\
278 & 1 \\
279 & 2 \\
280 & 0 \\
281 & 2 \\
282 & 2 \\
283 & 2 \\
284 & 1 \\
285 & 3 \\
286 & 1 \\
287 & 2 \\
288 & 2 \\
289 & 0 \\
290 & 1 \\
291 & 1 \\
292 & 0 \\
293 & 1 \\
294 & 0 \\
295 & 2 \\
296 & 1 \\
297 & 1 \\
298 & 2 \\
299 & 1 \\
300 & 0 \\
301 & 1 \\
302 & 1 \\
303 & 0 \\
304 & 1 \\
305 & 1 \\
306 & 3 \\
307 & 1 \\
308 & 0 \\
309 & 1 \\
310 & 1 \\
311 & 3 \\
312 & 0 \\
313 & 3 \\
314 & 4 \\
315 & 0 \\
316 & 3 \\
317 & 1 \\
318 & 3 \\
319 & 0 \\
320 & 1 \\
321 & 0 \\
322 & 2 \\
323 & 0 \\
324 & 1 \\
325 & 1 \\
326 & 1 \\
327 & 0 \\
328 & 2 \\
329 & 2 \\
330 & 2 \\
331 & 1 \\
332 & 0 \\
333 & 3 \\
334 & 0 \\
335 & 1 \\
336 & 1 \\
337 & 0 \\
338 & 0 \\
339 & 0 \\
340 & 2 \\
341 & 0 \\
342 & 1 \\
343 & 0 \\
344 & 0 \\
345 & 0 \\
346 & 0 \\
347 & 1 \\
348 & 2 \\
349 & 3 \\
350 & 1 \\
351 & 0 \\
352 & 1 \\
353 & 0 \\
354 & 1 \\
355 & 1 \\
356 & 2 \\
357 & 1 \\
358 & 1 \\
359 & 0 \\
360 & 2 \\
361 & 0 \\
362 & 3 \\
363 & 2 \\
364 & 0 \\
365 & 0 \\
366 & 2 \\
367 & 0 \\
368 & 1 \\
369 & 1 \\
370 & 1 \\
371 & 1 \\
372 & 0 \\
373 & 0 \\
374 & 1 \\
375 & 0 \\
376 & 0 \\
377 & 2 \\
378 & 0 \\
379 & 0 \\
380 & 0 \\
381 & 1 \\
382 & 0 \\
383 & 1 \\
384 & 0 \\
385 & 1 \\
386 & 1 \\
387 & 1 \\
388 & 2 \\
389 & 0 \\
390 & 1 \\
391 & 4 \\
392 & 1 \\
393 & 0 \\
394 & 1 \\
395 & 1 \\
396 & 0 \\
397 & 2 \\
398 & 1 \\
399 & 0 \\
400 & 0 \\
401 & 1 \\
402 & 0 \\
403 & 1 \\
404 & 0 \\
405 & 0 \\
406 & 1 \\
407 & 0 \\
408 & 1 \\
409 & 1 \\
410 & 0 \\
411 & 0 \\
412 & 1 \\
413 & 0 \\
414 & 0 \\
415 & 0 \\
416 & 0 \\
417 & 0 \\
418 & 0 \\
419 & 0 \\
420 & 0 \\
421 & 0 \\
422 & 0 \\
423 & 0 \\
424 & 0 \\
425 & 0 \\
426 & 0 \\
427 & 0 \\
428 & 0 \\
429 & 0 \\
430 & 0 \\
431 & 0 \\
432 & 0 \\
433 & 0 \\
434 & 0 \\
435 & 0 \\
436 & 0 \\
437 & 0 \\
438 & 0 \\
439 & 0 \\
440 & 0 \\
441 & 0 \\
442 & 0 \\
443 & 0 \\
444 & 0 \\
445 & 0 \\
446 & 0 \\
447 & 0 \\
448 & 0 \\
449 & 0 \\
450 & 0 \\
451 & 0 \\
452 & 0 \\
453 & 0 \\
454 & 0 \\
455 & 0 \\
456 & 0 \\
457 & 0 \\
458 & 0 \\
459 & 0 \\
460 & 0 \\
461 & 0 \\
462 & 0 \\
463 & 0 \\
464 & 0 \\
465 & 0 \\
466 & 0 \\
467 & 0 \\
468 & 0 \\
469 & 0 \\
470 & 0 \\
471 & 0 \\
472 & 0 \\
473 & 0 \\
474 & 0 \\
475 & 0 \\
476 & 0 \\
477 & 0 \\
478 & 0 \\
479 & 0 \\
480 & 0 \\
481 & 0 \\
482 & 0 \\
483 & 0 \\
484 & 0 \\
485 & 0 \\
486 & 0 \\
487 & 0 \\
488 & 0 \\
489 & 0 \\
490 & 0 \\
491 & 0 \\
492 & 0 \\
493 & 0 \\
494 & 0 \\
495 & 0 \\
496 & 0 \\
497 & 0 \\
498 & 0 \\
499 & 0 \\
500 & 0 \\
501 & 0 \\
502 & 0 \\
503 & 0 \\
504 & 0 \\
505 & 0 \\
506 & 0 \\
507 & 0 \\
508 & 0 \\
509 & 0 \\
510 & 0 \\
\end{longtable}


\printbibliography{}

\appendix
\newpage

\section{Anhang}
\label{sec:Anhang}
\subsection{Implementierung der Auswertungen}
Im folgenden ist der python-Code aufgeführt, der zur Auswertung der Messdaten verwendet wurde.
Verwendet wurden die Bibliotheken numpy \cite{numpy}, scipy \cite{scipy}, matplotlib \cite{matplotlib} und uncertainties \cite{uncertainties}.

% \lstinputlisting[basicstyle=\tiny, linerange={181-236}, breaklines=true,]{plot.py}
\lstinputlisting[basicstyle=\tiny, breaklines=true,]{plot.py}

\end{document}
