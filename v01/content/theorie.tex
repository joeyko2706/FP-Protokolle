\section{Zielsetzung}\label{sec:Zielsetzung}

Ziel dieses Versuches ist es die Lebensdauer kosmischer Myonen zu bestimmen, um dabei grundlegende Detektorkomponenten näher kennen zu lernen.

\section{Theorie}\label{sec:Theorie}

Die Informationen über die Teilchen werden der particle data group entnommen\cite{ParticleDataGroup:2024cfk}. \newline
Myonen sind Elementarteilchen aus der zweiten Generation der Fermionen und gehören zur Leptonenfamilie.
Die Myonen, die auf der Erde gemessen werden können, sind Sekundärteilchen, die bei der Wechselwirkung von hochenergetischer kosmischer Strahlung von zum Beispiel der Sonne mit den oberen Schichten der Erdatmosphäre entstehen.
Sie entstehen in einer Höhe von etwa $\SI{15}{\kilo\meter}$.
Trifft ein Proton auf ein Atom der Erdatmosphäre, entstehen bei der Kollision Sekundärteilchen.
Da Pionen und Kaonen die leichtesten Mesonen sind, entstehen diese bei der Kollision am häufigsten.
Diese wiederum zerfallen über die elektromagnetische Wechselwirkung in Myonen und Myon-Neutrinos via der folgenen Zerfallskanäle
\begin{align*}
    \pi^\pm&\rightarrow \mu^\pm\nu_\mu(\bar\nu_\mu),\\
    K^\pm&\rightarrow\mu^\pm\nu_\mu(\bar\nu_\mu).
\end{align*}
Um die Impulserhaltung einzuhalten, wird neben einem Myonen auch ein Myon-Neutrino $\nu_\mu$ (bei Zerfall eines $\pi^+$) oder Antineutrino $\bar\nu_\mu$ erzeugt (bei Zerfall eines $\pi^-$).
Selbiges gilt für die Kaonen. 
Die Myonen selber zerfallen in Elektronen und Elektron-Antineutrinos via
\begin{align*}
    \mu^\pm &\rightarrow e^\pm \nu_e(\bar\nu_e) \bar\nu_\mu (\nu_\mu).
\end{align*}
\newline
Die durchschnittliche Zeit, die ein Teilchen benötigt, um in andere Teilchen zu zerfallen wird (mittlere) Zerfallszeit $\tau$ genannt.
Sie folgt über das Zerfallsgesetz 
\begin{align}
    N(t)=N_0e^{-\lambda t},
\end{align}
wobei $N(t)$ die Anzahl der noch nicht zerfallenen Teilchen zum Zeitpunkt $t$ ist und $N_0$ die Anzahl der Teilchen zum Zeitpunkt $t=0$.
Die Zerfallskonstante $\lambda$ ist dabei ein Proportionalitätsfaktor, der für jedes Teilchen unterscheidlich ist.
Mithilfe des Zerfallsgesetzes kann nun die mittlere Lebensdauer $\tau$ eines Teilchens durch zeitliches Ableiten und umstellen bestimmt werden,
\begin{align}
    \frac{N(t)}{N_0}&=\lambda e^{-\lambda t}\text dt, \\
    \tau = \braket t &= \int_0^\infty t\lambda e^{-\lambda t}\text dt = \frac{1}{\lambda}.
\end{align}
Dass die Myonen auf der Erde auftreffenn lässt sich nach dem klassischen Modell der Physik nicht beschreiben, da die Lebensdauer der Myonen von $\SI{2.197}{\micro\second}$ zu kurz wäre, um die Erde zu erreichen,
\begin{align*}
    s = v\cdot t\approx \SI{3e8}{\meter\per\second}\cdot \SI{2.196e-6}{\second}\approx \SI{660}{\meter}.
\end{align*}
Die Myonen erreichen die Erde jedoch, da sie aufgrund ihrer hohen Geschwindigkeit eine Zeitdilatation erfahren, die nach der speziellen Relativitätstheorie von Albert Einstein beschrieben wird,
\begin{align*}
    \gamma = \frac{E}{m_\mu c^2} = \frac{10\cdot 10^9}{105,66\cdot 10^6}\approx94.7, \\
    \rightarrow s = vt\cdot\gamma\approx \SI{3e8}{\meter\per\second}\cdot \SI{2.196e-6}{\second} \cdot 94.7\approx \SI{62.4}{\kilo\meter}.
\end{align*}
Somit bestimmt sich die Wahrscheinlichkeit für ein Myon auf der Erde aufzutreffen zu
\begin{align*}
    P&=e^{-\sfrac{L}{s}}\ \text{mit}\ L=\SI{15}{\kilo\meter}, \\
    P_{\text{klassisch}} &= e^{-\sfrac{L}{\SI{660}{\meter}}}\approx 1.3\cdot 10^{-10}, \\
    P_{\text{relativistisch}} &= e^{-\sfrac{L}{\SI{62.4}{\kilo\meter}}}\approx 78.63\%.
\end{align*}